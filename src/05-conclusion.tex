\chapter{Conclusion}\label{sec:conclusion}

This report studies robo-advising algorithms for client's risk profile estimation and automated portfolio optimization. We given a comprehensive literature review on robo-advisors and a novel taxonomy of existing robo-advisors in the literature. Then, we focus on the setting where the robo-advisor follows a fixed interaction schedule and communicated with the clients on their biased risk aversion and attempts to make investment decisions based on mean-variance maximization. We elaborate on our numerical algorithm and its implementation and propose a method to estimate client's personalized parameters.

Robo-advisors are still emerging in the financial industry and the quantitative finance research community, and there are potential future research directions that we have not been able to explore due to time limitation. For example, one could try to propose an adaptive interaction schedule where the robo-advisor chooses to interact with the client only when it feels necessary to achieve a sweet spot in the trade-off between better estimation of risk aversion and less behavioral bias. Also, one could work on new estimation methods to provide more accurate estimations of the client's personalized parameters with fewer observations.