\chapter{Introduction}
% \section{Background}
% \section{Motivation and challenges}
% \section{Contributions}
% \section{Report organization}

Robo-advisors are automated investment platforms that utilize quantitative algorithms and data analytics to provide personalized portfolio management services to a broader range of investors. Since its introduction in the early 2000s, it has gained much popularity due to its lower cost, stronger personalization and better performance, and has especially been popularized since the financial crisis in 2008, when investor started to lose confidence in traditional financial institutions.

Due to its popularity in the financial industry and its benefits over conventional services provided by human advisors, there has been an emerging amount of attention drawn by robo-advisors in the quantitative finance research community as well. \citeA{beketov2018robo} survey existing robo-advisors on the market, study its recent emergence in the industry, and predict the trends of the robo-advising industry in the future. \citeA{tertilt2018advise} propose questionnaires to help robo-advisors evaluate the risk preferences of the client. \citeA{wang2020continuous} propose a reinforcement learning framework for robo-advisors to minimize risk given the client's target return value. More recently, there are an increasing number of works focusing on more personalized robo-advisors where the advisor estimates the client's risk aversion to provide more tailored investment suggestions. \citeA{wang2021robo} let the robo-advisor to observe the historical asset allocations made by the client to use inverse reinforcement learning to infer client risk aversion and optimize portfolio based on that. Similarly, \citeA{alsabah2021robo} allow the robo-advisor to not only observe client's asset allocation history, but to actively interact with the client during the robo-advising process as well. \citeA{capponi2022personalized} also take client's behavioral bias caused by his or her trend-chasing mindset into consideration.

In this report, we study robo-advisors under the setting of \citeA{capponi2022personalized}, which can be characterized as follows. First, the robo-advisor interacts with the client based on a pre-determined fixed schedule, where at each interaction, the client communicates its behaviorally biased risk aversion to the robo-advisor. The behavioral bias is due to the trend-chasing mindset of the client and is determined by the market performance since the previous interaction. Through interactions and observation of the environment, the robo-advisor models and estimates the client's risk aversion (with behavioral bias) and use that to manage portfolio to maximize the mean-variance objective function. We study this setting because as per previous discussions on existing settings, the setting of \citeA{capponi2022personalized} is the most comprehensive, which is personalized and interactive, and also takes behavioral bias into consideration.

Based on the setting of \citeA{capponi2022personalized}, we focus in particular on: (i) details on the numerical algorithm to calculate the optimal strategy and its implementation; (ii) estimation of client's personalized parameters. We focus on these two aspects particularly because they are not sufficiently discussed in the original paper. \citeA{capponi2022personalized} only briefly discuss the implementation of the algorithm with high level ideas and they assume that the robo-advisor has already known the personalized parameters of the client. Other than these two points, we also discuss various aspects of the robo-advisor via theoretical and empirical findings. To summarize, our contributions include the following.\begin{enumerate}
    \item We perform a comprehensive literature review on robo-advisors, including their history, advantages, typical workflow and existing solutions.
    \item We provide a novel taxonomy to categorize existing robo-advisor models based on three dimensions -- their optimization goal, client input and interaction model.
    \item We study in great detail the robo-advisor model of \citeA{capponi2022personalized}, especially its numerical algorithm and its implementation. To that end, we prove a novel theorem (Theorem \ref{thm:disc}) stating that the optimal strategy only depends on a low-dimensional tuple of state variables instead of full state history. This is key to algorithm implementation for discretization and backwards induction.
    \item We solve the novel problem of estimating client's personalized parameters, which is omitted by the original paper. We provide our method of moments estimators in Theorem \ref{thm:est}.
    \item We implement our robo-advisor algorithm and parameter estimators and run experiments with them. Empirical results demonstrate the correctness of our theoretical findings and we also discuss other properties of the robo-advisor via experiments.
\end{enumerate}

The rest of this report is organized as follows. Section \ref{sec:literature} surveys recent advances on robo-advisors in the literature and provides a taxonomy of selected robo-advising models. Section \ref{sec:approach} elaborates on our setting and the robo-advisor algorithm, with detailed discussions on the numerical computation of optimal strategies and its implementation, as well as estimation of client's personalized parameters. Section \ref{sec:exp} reports experimental results and includes discussions on important properties of our robo-advising algorithm and parameter estimators. At last, Section \ref{sec:conclusion} concludes our findings and gives a few potential research directions for further exploration.